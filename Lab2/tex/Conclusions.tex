
\section*{Concluzie}


Interfata grafica este numit sistemul de afisaj grafic-vizual pe un ecran, situat functional intre utilizator si dispozitive electronice cum ar fi computere, dispozitive personale de tip hand-held (playere MP3, playere media portabile, dispozitive de jucat), aparate electrocasnice si unele echipamente de birou. Pentru a prezenta toate informatile si actiunile disponibile, un GUI ofera pictograme si indicatori vizuali, in contrast cu interfetele bazate pe text, care ofera doar nume de comenzi (care trebuie tastate) sau navigatia text.

In aceasta lucrare de laborator mi-am dezvoltat abilitatile practice in GUI Development. Am creat propriul meu calculator, lucru care la prima vedere imi paruse complicat. Am intilnit probleme dar cu calm, documentindu-ma ale-am rezolvat. Cunoaterea interfetei grafice este importanta in dezvoltarea mea ca IT student. 


Interfata grafica trebuie sa fie:
• Consistenta: Pentru o anume operatie sa se foloseasca acelasi obiect vizual. Accesul la operatii similare se face prin aceleasi actiuni utilizator (mouse, tastatura) si folosind acelasi obiect vizual.
• Intuitiva: Interfata sa fie sugestiva, nefiind nevoie de documentatie sau de cursuri de pregatire.
• Atractiva: Interfata trebuie sa aiba caracteristici estetice care sa atraga utilizatorul. O interfata aglomerata va indeparta utilizatorii.
• Usor de utilizat: Operatiile simple trebuie sa se realizeze prin actiuni simple ale utilizatorului. Operatiile complexe trebuie sa se realizeze printr-un numar rezonabil de actiuni
utilizator.
• Usor de invatat: Orice actiune utilizator sa fie usor de memorat. Experienta acumulata in invatarea unor actiuni sa poata fi folosita la invatarea altor actiuni.  

\clearpage

      
