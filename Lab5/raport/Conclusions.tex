\section*{Concluzie}


In aceasta lucrare de laborator mi-am dezvoltat abilitatile practice in Mobile Development si spiritul de lucru in echipa. Am creat impreuna cu Nicolae o agenda de notite. Este a doua incercare a mea in crearea unui aplicatii mobile. IDE-ul folosit de noi a fost Android Studio.

Android Studio reprezinta un mediu integrat de dezvoltare aparut recent, dovedit a fi foarte eficient in dezvoltarea aplicatiilor, avind urmatoarele particularitati: functionarea dupa principiul WYSIWYG (What You See Is What You Get), posibilitatea de lucru cu elementele UI cu ajutorul functiei Dag-and-Drop; refactoring-ul codului; analizator static Lint; sabloane integrate a unei aplicatii Android; formarea aplicatiilor pe baza Gradle etc.  Desi destul de costisitoare, utilizind si mult timp pentru activarea emulatorului sau instalarea aplicatiei pe dispozitiv, Android Studio functioneaza dupa principiile generale de simplificare a etapelor de constructie a codului si interfetei, oferind o metoda relativ simplificata de formare a aplicatiei.


Indiferent ca este vorba despre un serviciu de dezvoltare aplicatii pentru Android sau iOS, crearea unei aplicatii a ajuns o necesitate pentru orice tip de afacere, aplictiile mobile corect realizate lasind intotdeauna o impresie de incredere si profesionalism. O aplicatie mobila trebuie vazuta ca o legatura intre comerciant si client, comerciantului fiindu-i mai ușor sa urmareasca comportamentul clientului sau sa ii prezinte ofertele intr-un mod personalizat.

\clearpage
